\subsection{Permanentmagneterregte Synchronmotoren} \label{sec:pmsm}

\blindtext{}

% https://de.mathworks.com/help/autoblks/ref/interiorpmsm.html
\vspace{0.5cm}
\begin{figure}[h]
    \centering
    \includegraphics[width=0.65\textwidth]{media/pic.png}
    \caption{PMSM mit Polpaarzahl 1~\cite{MathWorksPMSM}}
    \label{fig:PMSM_dq}
\end{figure}

\newpage{}

\blindtext{}

\vspace{0.3cm}

% Clarke-Transformation
\begin{formel}
    \begin{equation}
        \label{eq:clarke_alpha}
        I_\alpha = \frac{2}{3} \left(I_a - \frac{I_b + I_c}{2} \right)
    \end{equation}
    \caption[Clarke-Transformation $I_\alpha$~\protect{\cite[S.~309]{clarkeCircuitAnalysisAC1943}}]{}
\end{formel}

\vspace{-1cm}

\begin{formel}
\begin{equation}
    \label{eq:clarke_beta}
    I_\beta = \frac{1}{\sqrt{3}} (I_b-I_c)
\end{equation}
    \caption[Clarke-Transformation $I_\beta$~\protect{\cite[S.~309]{clarkeCircuitAnalysisAC1943}}]{}
\end{formel}

\vspace{-1cm}

\begin{formel}
    \begin{equation}
        \label{eq:clarke_mat}
        \begin{bmatrix}
            I_\alpha(t) \\
            I_\beta(t)
        \end{bmatrix} 
        = 
        \begin{bmatrix}
            1                   &   0 \\
            \frac{1}{\sqrt{3}}  &   \frac{2}{\sqrt{3}}
        \end{bmatrix}
        \begin{bmatrix}
            I_U \\
            I_V
        \end{bmatrix}
    \end{equation}
    \caption[Clarke-Transformation in Matrixschreibweise]{}
\end{formel}

\vspace{-0.8cm}
\blindtext{}
\cite{MathWorksPark}
\cite{clarkeCircuitAnalysisAC1943}
\cite{bolteElektrischeMaschinenGrundlagen2018}


\vspace{0.3cm}

% Park-Transformation
\begin{formel}
    \begin{equation}
        \label{eq:park}
        \begin{bmatrix}
            I_d \\
            I_q
        \end{bmatrix}
        =
        \begin{bmatrix}
            \cos(\theta)  &   \sin(\theta) \\
            -\sin(\theta) &   \cos(\theta)
        \end{bmatrix}
        \begin{bmatrix}
            I_\alpha \\
            I_\beta
        \end{bmatrix}
    \end{equation}
    \caption[Park-Transformation~\cite{MathWorksPark}]{}
\end{formel}

\vspace{-0.8cm}
\blindtext{}

\vspace{0.3cm}

% Drehmomentgleichung
\begin{formel}
    \begin{equation}
        \label{eq:Moment}
        M = \frac{3}{2} * pp * (\Psi_{d} * I_{q} - \Psi_{q} * I_{d})
    \end{equation}
    \caption[Drehmomentgleichung aus magnetischem Fluss~\protect{\cite[S.~507]{bolteElektrischeMaschinenGrundlagen2018}}]{}
\end{formel}

\vspace{-1cm}

\begin{formel}
    \begin{equation}
        \label{eq:Psi-L}
        \Psi = I_d * L_d
    \end{equation}
    \caption[Zusammenhang magnetischer Fluss, Strom und Induktivität~\protect{\cite[S.~43]{bolteElektrischeMaschinenGrundlagen2018}}]{}
\end{formel}

\vspace{-1cm}

\begin{formel}
    \begin{equation}
        \label{eq:Moment-L}
        M = \frac{3}{2} * pp * I_d * I_q * (L_d - L_q)
    \end{equation}
    \caption[Drehmomentgleichung mit Induktivität]{}
\end{formel}

\vspace{-0.8cm}

\blindtext{}

\vspace{0.3cm}

\begin{figure}[h]
    \centering
    \includegraphics[width=0.7\textwidth]{media/pic.png}
    \caption{Rotor mit vergrabenen (links) und Oberflächenmagneten (rechts)~\cite[S. 370]{schroederElektrischeAntriebeGrundlagen2021}}
    \label{fig:Magnete}
\end{figure}

\newpage{}
\blindtext{}